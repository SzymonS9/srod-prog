\documentclass{report}
\usepackage{polski}
\usepackage[utf8]{inputenc}

\title{Giełda}
\author{Szymon Stefański}
\begin{document}
\maketitle
\tableofcontents
\newpage
\chapter{Wstęp}
\section{Podstawowe pojęcia}
\subsection{Czym jest giełda?}
\textbf{Giełda}\cite{pa} – organizowane w ustalonym miejscu i czasie spotkania handlowe, na których sprzedawane są ściśle określone towary po cenach ogłoszonych w codziennych notowaniach. Transakcje\footnote{ czynność zachodząca między sprzedającym i kupującym, która ma na celu wymianę towaru lub usługi. Transakcja finansowa kończy się umową sprzedaży. } na giełdach zawierane są zgodnie z obowiązującym regulaminem, między członkami giełdy pośredniczącymi w zawieraniu transakcji.
\subsection{Organizacja giełdy}
Giełda jest kontrolowana przez państwo, które udziela koncesji na jej działanie i określa sposób jej działania.
Giełdy można podzielić na dwie kategorie:
\begin{itemize}
\item giełda państwowa
\item giełda korporacyjna.
\end{itemize}
W pierwszym przypadku państwo jest jedynym właścicielem giełdy i decyduje o wszystkim co jest związane z jej działaniem. W drugim przypadku giełdę tworzymy w oparciu o normy prawa handlowego.Rynek papierów wartościowych, który jest tworzony przez banki, domy maklerskie, fundusze inwestycyjne i inne instytucje jest instytucją, na którą państwo nie ma bezpośredniego wpływu.
\newline Do prawidłowego funkcjonowania giełdy muszą zostać spełnione pewne warunki:
\begin{itemize}
\item określanie uczestników giełdy,
\item stworzenie statutu i regulaminu,
\item ustalić strukturę organizacyjną,
\item dysponować odpowiednim wyposażeniem materialnym.
\end{itemize}
\section {Rodzaje giełd}
\newpage
\chapter{}
\section{Czynniki sukcesu na giełdzie}
Święty Graal inwestowania nie istnieje, a nawet jeżeli istnieje to niczego inwestorowi nie da jeżeli nie posiada on w sobie opanowania emocji, dyscypliny i konsekwencji. Nawet najlepsze strategie inwestycyjne, prosto z siedziby głównej banku Goldman Sachs, nic nam nie dadzą, jeżeli nie będziemy ich konsekwentnie stosować.\cite{fib}
\newpage
\chapter{}
\section{pierwszy podrozdział Trzeciego}
\begin{thebibliography}{99}
\addcontentsline{toc}{chapter}{Bibliografia}
\bibitem{pa}pojęcie giełda
https://pl.wikipedia.org/wiki/Giełda
\bibitem{fib}czynniki sukcesu
https://strefainwestorow.pl/wykres\_tygodnia/wyniki\_badania\_czynniki\_sukcesu\_na\_gieldzie
\end{thebibliography}
\end{document}